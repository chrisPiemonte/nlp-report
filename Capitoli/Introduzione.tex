% !TEX encoding = UTF-8
% !TEX TS-program = pdflatex
% !TEX root = ../Tesi.tex
% !TEX spellcheck = it-IT

%*******************************************************
% Introduzione
%*******************************************************
\cleardoublepage
\chapter*{Introduction}

The purpose of this document is to provide an overview of the state of the art in the area of the User Experience in the financial field, analyzing the reference literature.

Following the title of the research project, it was decided to divide the technical report into three sections: the first section will provide an historical overview of the main concepts.

In the second part techniques to design the user experience of the chatbots will be discussed. The main design patterns and guidelines will be exposed.

In the third part the attention will be focused on the added value that Chatbots can bring in the financial field, which Design Patterns must be considered and how they can affect the overall User Experience. Examples of experiments presented in literature will be presented together with a categorization of the types of Conversational agents.

Finally, the report will end with a critical analysis, emphasizing the elements that characterize the project with respect to the current state of the art.

%Lo scopo di questo documento è di fornire una panoramica dello stato dell'arte nell'area della modellazione utente, analizzando la letteratura di riferimento, enfatizzando gli elementi caratterizzanti del progetto di ricerca rispetto allo stato dell'arte e sottolineando le analogie con alcune metodologie già presentate in letteratura. Seguendo il titolo del progetto di ricerca, si è deciso di suddividere il rapporto tecnico in tre sezioni: nella prima saranno analizzati i principali lavori a stato dell’arte nell’area dello user modeling, fornendo una panoramica storica dell’area di ricerca e mostrando le principali tecniche per l’acquisizione dati e la modellazione utente. Successivamente si entrerà nel merito delle possibili sorgenti dati cui attingere per la costruzione dei modelli utente e si fornirà un’analisi critica delle stesse. Nella seconda parte l’attenzione sarà focalizzata sui meccanismi per la rappresentazione semantica dell’informazione: in questo caso si introdurranno i concetti di base dell’area fornendo una panoramica sulle tecniche di semantic natural language processing e sulle tecniche a stato dell’arte per la rappresentazione dei profili, fornendo anche una serie di riferimenti su lavori già presentati in letteratura. Nell’ultima parte, infine, si studierà il tema della modellazione olistica dell’individuo, analizzando metodologie a stato dell’arte per l’aggregazione, il riuso e la mediazione di modelli utente. Infine si concluderà il report illustrando le principali soluzioni tecnologico-architetturali per la modellazione utente ed infine analizzando criticamente quanto presentato, enfatizzando gli elementi caratterizzanti del progetto rispetto allo stato dell’arte attuale.
